\documentclass{article}

\usepackage{fancyhdr}
\usepackage{extramarks}
\usepackage{amsmath}
\usepackage{amsthm}
\usepackage{amsfonts}
\usepackage{tikz}
\usepackage[plain]{algorithm}
\usepackage{algpseudocode}
\usepackage{enumerate}
\usepackage{tikz}

\usetikzlibrary{automata,positioning}

%
% Basic Document Settings
%  

\topmargin=-0.45in
\evensidemargin=0in
\oddsidemargin=0in
\textwidth=6.5in
\textheight=9.0in
\headsep=0.25in

\linespread{1.1}

\pagestyle{fancy}
\lhead{\hmwkAuthorName}
\chead{\hmwkClass : \hmwkTitle}
\rhead{\firstxmark}
\lfoot{\lastxmark}
\cfoot{\thepage}

\renewcommand\headrulewidth{0.4pt}
\renewcommand\footrulewidth{0.4pt}

\setlength\parindent{0pt}

%
% Create Problem Sections
%

\newcommand{\enterProblemHeader}[1]{
    \nobreak\extramarks{}{Problem \arabic{#1} continued on next page\ldots}\nobreak{}
    \nobreak\extramarks{Problem \arabic{#1} (continued)}{Problem \arabic{#1} continued on next page\ldots}\nobreak{}
}

\newcommand{\exitProblemHeader}[1]{
    \nobreak\extramarks{Problem \arabic{#1} (continued)}{Problem \arabic{#1} continued on next page\ldots}\nobreak{}
    \stepcounter{#1}
    \nobreak\extramarks{Problem \arabic{#1}}{}\nobreak{}
}

\newcommand*\circled[1]{\tikz[baseline=(char.base)]{
		\node[shape=circle,draw,inner sep=2pt] (char) {#1};}}


\setcounter{secnumdepth}{0}
\newcounter{partCounter}
\newcounter{homeworkProblemCounter}
\setcounter{homeworkProblemCounter}{1}
\nobreak\extramarks{Problem \arabic{homeworkProblemCounter}}{}\nobreak{}

%
% Homework Problem Environment
%
% This environment takes an optional argument. When given, it will adjust the
% problem counter. This is useful for when the problems given for your
% assignment aren't sequential. See the last 3 problems of this template for an
% example.
%

\newenvironment{homeworkProblem}[1][-1]{
    \ifnum#1>0
        \setcounter{homeworkProblemCounter}{#1}
    \fi
    \section{Problem \arabic{homeworkProblemCounter}}
    \setcounter{partCounter}{1}
    \enterProblemHeader{homeworkProblemCounter}
}{
    \exitProblemHeader{homeworkProblemCounter}
}

%
% Homework Details
%   - Title
%   - Class
%   - Due date
%   - Name
%   - Student ID

\newcommand{\hmwkTitle}{Homework\ \#03}
\newcommand{\hmwkClass}{Probability \& Statistics for EECS}
\newcommand{\hmwkDueDate}{March 05, 2023}
\newcommand{\hmwkAuthorName}{Zhou Shouchen}
\newcommand{\hmwkAuthorID}{2021533042}


%
% Title Page
%

\title{
    \vspace{2in}
    \textmd{\textbf{\hmwkClass:\\  \hmwkTitle}}\\
    \normalsize\vspace{0.1in}\small{Due\ on\ \hmwkDueDate\ at 23:59}\\
	\vspace{4in}
}

\author{
	Name: \textbf{\hmwkAuthorName} \\
	Student ID: \hmwkAuthorID}
\date{}

\renewcommand{\part}[1]{\textbf{\large Part \Alph{partCounter}}\stepcounter{partCounter}\\}

%
% Various Helper Commands
%

% Useful for algorithms
\newcommand{\alg}[1]{\textsc{\bfseries \footnotesize #1}}
% For derivatives
\newcommand{\deriv}[1]{\frac{\mathrm{d}}{\mathrm{d}x} (#1)}
% For partial derivatives
\newcommand{\pderiv}[2]{\frac{\partial}{\partial #1} (#2)}
% Integral dx
\newcommand{\dx}{\mathrm{d}x}
% Alias for the Solution section header
\newcommand{\solution}{\textbf{\large Solution}}
% Probability commands: Expectation, Variance, Covariance, Bias
\newcommand{\E}{\mathrm{E}}
\newcommand{\Var}{\mathrm{Var}}
\newcommand{\Cov}{\mathrm{Cov}}
\newcommand{\Bias}{\mathrm{Bias}}

\begin{document}

\maketitle

\pagebreak

\begin{homeworkProblem}[1]
    From observation, we could totally discover $4$ diffenent end-to-end path that make it functional.\\
    And they are:\\
    $A_1 : $ "device $1, 3$ are functional".\\
    $A_2 : $ "device $1, 5, 4$ are functional".\\
    $A_3 : $ "device $2, 4$ are functional".\\
    $A_4 : $ "device $2, 5, 3$ are functional".\\
    Let $B = \bigcup\limits_{i=1}^4 A_i$.\\
    And the event $B$ is what we want, i.e. all possible situation that make it functional.\\

    And since each device is independent, so $P(A_{j_1}\cap \cdots \cap A_{j_i})=p^n$, where $n$ is the number of diffenent devices that are mentioned in $A_{j_1},\cdots,A_{j_i}$.\\
    So $$P(B) = P(\bigcup\limits_{i=1}^4 A_i)=\sum_{i=1}^{4}(-1)^{i+1}P(A_{j_1}\cap \cdots \cap A_{j_i})$$
    $$=[P(A_1)+P(A_2)+P(A_3)+P(A_4)]$$
    $$-[P(A_1,A_2)+P(A_1,A_3)+P(A_1,A_4)+P(A_2,A_3)+P(A_2,A_4)+P(A_3,A_4)]$$
    $$+[P(A_1,A_2,A_3)+P(A_1,A_2,A_4)+P(A_1,A_3,A_4)+P(A_2,A_3,A_4)]$$
    $$-[P(A_1,A_2,A_3,A_4)]$$
    $$=(p^2+p^3+p^2+p^3)-(p^4+p^4+p^4+p^4+p^5+p^4)+(p^5+p^5+p^5+p^5)-(p^5)$$
    $$=2p^5-5p^4+2p^3+2p^2$$

    So above all, the probability that the system functions is that $2p^5-5p^4+2p^3+2p^2$.\\

\end{homeworkProblem}

\newpage

\begin{homeworkProblem}[2]

    (a)Intuitively, there are three kinds of students that ate admitted in the club, and the three types are:\\
    
    1. The student is both good at Genshin Impact and Apex.\\
    2. The student is only good at Genshin Impact, but not good at Apex.\\
    3. The student is only good at Apex, but not good at Genshin Impact.\\
    
    Although it is said that a student is good at Genshin Impact or Apex is independent, but actually, considerring the
    result of admitted students.\\
    If we conditioning on a student is good at Apex, then the students that are from type 2 are not considered, i.e. the students only good at Genshin Impact are not considered. So good at Apex decrease the probability that is good at Genshin Impact.\\
    Similarly, if we conditioning on a student is good at Genshin Impact, then the students that are from type 3 are not considered, i.e. the students only good at Apex are not considered. So good at Genshin Impact decrease the probability that is good at Apex.\\
    So actually, the student is good at Genshin Impact or Apex are negatively associated, conditioning on being good at Apex decreases the chance of being good at Genshin Impact.\\
    
    (b) Since $C = A \cup B$, so $B\cap C = B, A\cap C = A$.\\
    So $ P(A|B\cap C) = P(A|B) $,\\
    since $A$ and $B$ are independent, so $P(A|B) = P(A)$.\\
    i.e. $ P(A|B\cap C) = P(A) $.\\
    And from the definition of conditional probability, we can get that $P(A|C)=\dfrac{P(A\cap C)}{P(C)} = \dfrac{P(A)}{P(A\cup B)}$.\\
    And since we are given that $P(A\cup B) < 1$, so $P(A|C) = \dfrac{P(A)}{P(A\cup B)} > P(A) = P(A|B\cap C)$.\\
    So above all, $P(A|B\cap C) < P(A|C)$.

\end{homeworkProblem}

\newpage

\begin{homeworkProblem}[3]

(a) for $p_0$, before we roll the die, the total is $0$, so $p_0 = 1$.\\
And it is impossible to have the negative total, so $p_k = 0$, for $k < 0$.\\
And for a total $n$, it has equal probability to come from $n-1,n-2,\cdots,n-6$, i.e. roll a die with the point $1,2,\cdots,6$ in this step correspondingly.\\
So we have $p_n = \dfrac{1}{6}(p_{n-1} + p_{n-2} + p_{n-3} + p_{n-4} + p_{n-5} + p_{n-6})$.\\
So above all, $p_0 = 1, p_k = 0 ,\forall k < 0$,\\
and $p_n = \dfrac{1}{6}(p_{n-1} + p_{n-2} + p_{n-3} + p_{n-4} + p_{n-5} + p_{n-6})$.\\

(b) from the recursive equation and the initial conditions above, we could calculate that\\
$p_1 = \dfrac{1}{6}, p_2 = \dfrac{1}{6}(p_0 + p) = \dfrac{1}{6}(1+\dfrac{1}{6})$,
$p_3 = \dfrac{1}{6}(p_0 + p_1 + p_2) = \dfrac{1}{6}(p_0 + p_1) + \dfrac{1}{6}p_2 = (1 + \dfrac{1}{6})p_2 = \dfrac{1}{6}(1+\dfrac{1}{6})^2$,\\
$p_4 = \dfrac{1}{6}(p_0 + p_1 + p_2 + p_3) = \dfrac{1}{6}(p_0 + p_1 + p_2) + \dfrac{1}{6}p_3 = (1 + \dfrac{1}{6})p_3 = \dfrac{1}{6}(1+\dfrac{1}{6})^3$,\\
$p_5 = \dfrac{1}{6}(p_0 + p_1 + p_2 + p_3 + p_4) = \dfrac{1}{6}(p_0 + p_1 + p_2 + p_3) + \dfrac{1}{6}p_4 = (1 + \dfrac{1}{6})p_4 = \dfrac{1}{6}(1+\dfrac{1}{6})^4$,\\
$p_6 = \dfrac{1}{6}(p_0 + p_1 + p_2 + p_3 + p_4 + p_5) = \dfrac{1}{6}(p_0 + p_1 + p_2 + p_3 + p_4) + \dfrac{1}{6}p_5 = (1 + \dfrac{1}{6})p_5 = \dfrac{1}{6}(1+\dfrac{1}{6})^5$,\\
$p_7 = \dfrac{1}{6}(p_1 + p_2 + p_3 + p_4 + p_5 + p_6) = \dfrac{1}{6}(p_0 + p_1 + p_2 + p_3 + p_4 + p_5) + \dfrac{1}{6}p_6 - \dfrac{1}{6}p_0= (1 + \dfrac{1}{6})p_6 - \dfrac{1}{6}p_0= \dfrac{1}{6}(1+\dfrac{1}{6})^6 - \dfrac{1}{6}$\\

So above all, $p_7 = \dfrac{1}{6}(1+\dfrac{1}{6})^6 - \dfrac{1}{6} \approx 0.25360439529$\\

(c) Intuitively, the average point we can get after roll a die is $\dfrac{\sum\limits_{i=1}^{6}i}{6}=\dfrac{21}{6}=\dfrac{7}{2}$.
And we can intuitively regard it as we may reach $2$ different points in every $7$ points. So the possibility that we can reach a certain
point when $n\to \infty$ is $\dfrac{\# lands\ we\ can\ reach}{\# all\ possible\ points} = \dfrac{2}{7} = \dfrac{1}{3.5}$.\\
So above all, when $n\to \infty, p_n \to \dfrac{1}{3.5} = \dfrac{2}{7}$.\\

\end{homeworkProblem}

\newpage

\begin{homeworkProblem}[4]

(a) for $p_{i,j}$, it is the probability that bought $i$ toys, have exactly $j$ types.\\
So there are total $2$ different situations.\\
1. The $i$-th toy's type is in one of the first $i-1$ toys. Which means that after buying the $(i-1)$-th toy, there already exist $j$ types of toys.\\
    So for the $i$-th toy, it has $j$ available choices. So the probability of choosing the $i$-th toy is that $p = \dfrac{\# available\ choices}{\# all\ possible\ choices} = \dfrac{j}{n}$.
    And since buying the $i$-th toy is indepentant with buying the first $(i-1)$ toys, so for this situation, the total possibility is $\dfrac{j}{n}p_{i-1,j}$.\\

2. The $i$-th toy's type diffenent with all of the first $i-1$ toys. Which means that after buying the $(i-1)$-th toy, there already exist $j-1$ types of toys.\\
So for the $i$-th toy, it has $n-(j-1) = n-j+1$ available choices. So the probability of choosing the $i$-th toy is that $p = \dfrac{\# available\ choices}{\# all\ possible\ choices} = \dfrac{n-j+1}{n}$.
And since buying the $i$-th toy is indepentant with buying the first $(i-1)$ toys, so for this situation, the total possibility is $\dfrac{n-j+1}{n}p_{i-1,j-1}$.\\

So mergeing the two situations, we can get that $p_{i,j} = \dfrac{n-j+1}{n}p_{i-1,j-1} + \dfrac{j}{n}p_{i-1,j}$.\\
And this equalion is for the valid situations, which means that the types of toys is no bigger than the number of toys, i.e. $i\geq 2, 0\leq j\leq n, j\leq i$.
And for the other invalid situations, the probability is just $0$ since they are invalid, impoosible to happen.\\

So above all, for $i\geq 2$ and $1\leq j\leq n$,\\
if $j\leq i$, $p_{i,j} = \dfrac{n-j+1}{n}p_{i-1,j-1} + \dfrac{j}{n}p_{i-1,j}$,\\
if $j > i$, $p_{i,j} = 0$.\\

(b) To calculate $p_{i,j}$, we can use the recursive equation $p_{i,j} = \dfrac{n-j+1}{n}p_{i-1,j-1} + \dfrac{j}{n}p_{i-1,j}$ to calculate every $p_{i,j}$.\\
As for the boundary condition, we just take $p_{1,1} = 1$.\\
Since the types of toys is impossible to be less than $0$, or bigger than the number of toys, so all invalid term: $j = 0$ or $j >i$, $p_{i,j} = 0$.\\
Then we can just calculate them.\\
Firstly calculate $p_{2,1}$, then $p_{2,2}$.\\
After that, calculate $p_{3,1}$, then $p_{3,2}$, then $p_{3,3}$.\\
...\\
After calculating $p_{i,1}, p_{i,2},\cdots, p_{i,i}$, we can use the calculated result to get 
$p_{i+1,1}, p_{i+1,2},\cdots, p_{i+1,i+1}$.\\

So above all, with the recursion equation and the order, the boundary conditions, we can calculate all $p_{i,j}$ that are needed.\\

\end{homeworkProblem}

\newpage

\begin{homeworkProblem}[5]

(a) We can only consider the first step, and the other steps can be recursively gotten.\\
Let \\
$p_k$ : the probability that the drunk ever reaches the value $k$.\\
$A_k$ :  the drunk ever arrived $k$.\\
$B_l$ : the drunk's first step is to the left.\\
$B_r$ : the drunk's first step is to the right.\\

From the given imformation, we know that $P(B_l) = q, P(B_r) = p$.\\
Since the drunk started at the origin, so $p_0 = 1$, and we only consider the value $k\geq 0$.\\
And considering the first step, according to LOTP, $p_k = P(A_k) = P(A_k|B_l)P(B_l) + P(A_k|B_r)P(B_r)$.\\
If the drunk's first step is to the left. then he is on the value of $-1$, and we want to know the probability that he ever moved to $k$,
which is exactly the same as he is now on the value of $0$, and we want to know the probability that he ever move to $k+1$.
So it means that $P(A_k|B_l) = P(A_{k+1}) = p_{k+1}$.\\
Similarly, $P(A_k|B_r) = P(A_{k-1}) = p_{k-1}$.\\
So $p_k = P(A_{k+1})\cdot q + P(A_{k-1})\cdot p = p_{k+1}\cdot q + p_{k-1}\cdot p$.\\

So above all, the recursive equation is $p_k = p_{k+1}\cdot q + p_{k-1}\cdot p$, $k\geq 0$, $p_0 = 1$\\

(b) From the class, we had learned about the Gambler's Ruin, which has a lot of similarities with this problem.
$$(p_i)'= \begin{cases}\dfrac{1-\left(\dfrac{q}{p}\right)^i}{1-\left(\dfrac{q}{p}\right)^N} & \text { if } p \neq 1 / 2 \\ \dfrac{i}{N} & \text { if } p=1 / 2 .\end{cases}$$
Where $(p_i)'$ is the possibility to win in the Gambler's Ruin.\\
And let $p_{i,k}$ means that the probability that the drunk man had moved to the value $k$ before moving to the value $-i$,
so it is clear that $$p_k = \lim_{i\to +\infty}p_{i,k}$$
And $p_{i,k}$ can be transfer to the Gambler's Ruin, i.e. started at the value of $0$, and lose if move to the value $-i$, win if move to the value $k$.\\
If we plus $i$ to all valus, then it is exactly the same as the  Gambler's Ruin, the man started at $i$, and $N = i + k$.\\
So with the conclusion of the Gambler's Ruin,
$$p_{i,k}= \begin{cases}\dfrac{1-\left(\dfrac{q}{p}\right)^i}{1-\left(\dfrac{q}{p}\right)^{i+k}} & \text { if } p \neq 1 / 2 \\ \dfrac{i}{i+k} & \text { if } p=1 / 2 .\end{cases}$$

\begin{itemize}
    \item if $p < \dfrac{1}{2}$, then $\dfrac{q}{p} > 1$,\\
    so $p_k = \lim_{i\to +\infty} \dfrac{1-\left(\dfrac{q}{p}\right)^i}{1-\left(\dfrac{q}{p}\right)^{i+k}} = \dfrac{1}{(\dfrac{q}{p})^k} = (\dfrac{p}{q})^k$

    \item if $p = \dfrac{1}{2}$, then $\dfrac{q}{p} = 1$,\\
    so $p_k =  \lim_{i\to +\infty} \dfrac{i}{i+k} = 1$

    \item if $p > \dfrac{1}{2}$, then $\dfrac{q}{p} < 1$,\\
    so $p_k = \lim_{i\to +\infty} \dfrac{1-\left(\dfrac{q}{p}\right)^i}{1-\left(\dfrac{q}{p}\right)^{i+k}} = 1$
\end{itemize}    

So above all, if $p < \dfrac{1}{2}, p_k = (\dfrac{p}{q})^k$,\\
if $p = \dfrac{1}{2}, p_k = 1$,\\
if $p > \dfrac{1}{2}, p_k = 1$.\\

\hrule
(b) had some discussions with others, and with a little bit help from the internet.\\

\end{homeworkProblem}

\newpage

\end{document}
